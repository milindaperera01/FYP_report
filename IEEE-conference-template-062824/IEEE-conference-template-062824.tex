\documentclass[conference]{IEEEtran}
\IEEEoverridecommandlockouts
% The preceding line is only needed to identify funding in the first footnote. If that is unneeded, please comment it out.
%Template version as of 6/27/2024

\usepackage{cite}
\usepackage{amsmath,amssymb,amsfonts}
\usepackage{algorithmic}
\usepackage{graphicx}
\usepackage{textcomp}
\usepackage{xcolor}
\def\BibTeX{{\rm B\kern-.05em{\sc i\kern-.025em b}\kern-.08em
    T\kern-.1667em\lower.7ex\hbox{E}\kern-.125emX}}
\begin{document}

\title{EEG-Based Stress Detection Using Deep Learning Techniques : A Survey\\
\thanks{Identify applicable funding agency here. If none, delete this.}
}

\author{\IEEEauthorblockN{Perera K.M.S.
\\Student}
\IEEEauthorblockA{\textit{Dept of Computer Engineering} \\
\textit{University of Peredeniya}\\
Sri Lanka\\
milindaperera14996@mail.com}
\and
\IEEEauthorblockN{ Moda malisha
\\Student}
\IEEEauthorblockA{\textit{Dept of Computer Engineering} \\
\textit{University of Peredeniya}\\
Sri Lanka\\
milindaperera14996@mail.com}
\and
\IEEEauthorblockN{ Moda nadeeka
\\Student}
\IEEEauthorblockA{\textit{Dept of Computer Engineering} \\
\textit{University of Peredeniya}\\
Sri Lanka\\
milindaperera14996@mail.com}
}

\maketitle

\begin{abstract}
Distinguishing between simple physiological arousal and the maladaptive state of true stress is crucial. 
This harmful condition arises when environmental demands exceed an organism's coping capacity, impairing physiological 
recovery. Due to its widespread negative impact, early and accurate detection of maladaptive stress is vital 
for protecting personal, social, and economic wellbeing in today’s fast-paced world. Electroencephalography (EEG), 
as a primary method for monitoring neural activity, has shown significant promise in identifying the distinct 
brain states associated with this stress. Modern deep learning architectures provide a powerful approach by 
automatically extracting meaningful patterns and hierarchical features from complex EEG data. This comprehensive 
survey offers researchers, practitioners, and technology enthusiasts a definitive overview of current advancements 
and highlights the future directions of EEG-based stress detection using deep learning.
\end{abstract}

\begin{IEEEkeywords}
component, formatting, style, styling, insert.
\end{IEEEkeywords}

\section{Introduction}
Stress, a biological response to internal or external stimuli, plays a critical role in exacerbating numerous pathological conditions\cite{yaribeygi2017impact}. 
It is generally classified as acute (short-term) or chronic (long-term), with unresolved chronic stress leading to serious health consequences 
by impairing the immune system and disrupting neuroendocrine functions. This persistent state is associated with a broad spectrum of illnesses, 
including cardiovascular disorders, diabetes, mental health conditions such as depression, and structural brain changes that negatively affect 
cognition and memory\cite{mariotti2015effects}. Recognizing the profound impact of stress, research has shifted toward identifying objective physiological markers 
beyond subjective assessments. Common approaches monitor autonomic nervous system responses through peripheral signals like Heart Rate Variability 
(HRV) from Electrocardiography (ECG), Galvanic Skin Response (GSR), and skin temperature\cite{kyrou2024deep}\cite{vanitha2013hybrid}\cite{herborn2015skin}. However, attention has increasingly turned to the brain’s 
electrical activity as the origin of the stress response, positioning Electroencephalogram (EEG) signals as a valuable, non-invasive, real-time tool 
for assessing neural dynamics. Traditional machine learning (ML) methods, especially Support Vector Machines (SVMs), have shown promise in 
classifying stress states from EEG data. Yet, these methods often depend heavily on manual, time-intensive feature extraction. In contrast, deep 
learning (DL) architectures offer a robust alternative by automatically learning meaningful features from complex, high-dimensional EEG signals. 
Models such as Convolutional Neural Networks (CNNs) excel at capturing spatial patterns, while Long Short-Term Memory (LSTM) networks effectively 
model temporal dependencies. Hybrid CNN-LSTM models are increasingly demonstrating superior classification accuracy by integrating both spatial and 
temporal information. This survey aims to deliver a thorough review of EEG-based stress detection using advanced deep learning techniques, analyzing 
current methodologies and performance trends, identifying key challenges, and exploring promising directions for future research.

\section{BACKGROUND}

\subsection{Stress}\label{AA}
Stress is a psychophysiological response of the human body to internal or external stressors, manifesting in physical, mental, or emotional forms. 
It is triggered when an individual perceives a situation as challenging, threatening, or demanding. 
Stressors may be biological, developmental, psychological, socio-cultural, or environmental, and even positive events can disrupt 
homeostasis, the body's internal balance. Stress is generally classified into two main categories: short-term (episodic) and long-term stress (chronic). 
Short-term stress arises from specific tasks or situations such as examinations, deadlines, or sudden pressures. Repeated exposure to such stressors 
may result in episodic stress, often linked to anxiety and hypertension \cite{kyrou2024deep}. Prolonged exposure leads to chronic stress, which can cause depression, 
psychological disorders, and severe physical illnesses. Chronic stress disrupts the nervous system, 
impairs functional capacity, and adversely affects daily life \cite{kyrou2024deep}.

Assessing stress is difficult because people react differently to the same stressor, and the same individual may react differently at different times. Clinical and 
psychological tools such as the Perceived Stress Scale are used for evaluation, but these survey-based methods are better suited for long-term psychological conditions and
 may not capture real-time stress fluctuations \cite{hafeez2024eeg}. In addition, physiological methods use bio-signals like ECG-based HRV, speech patterns, and galvanic skin response, all 
 of which change with mental stress. More recently, EEG has gained attention as a non-invasive and reliable signal for detecting stress-related brain activity, making it a key 
 focus in deep-learning-based stress detection research.

Today, stress has become a major public health concern, intensified by fast-paced lifestyles, heavy workloads, and increasing academic pressure on students. Prolonged or 
unmanaged stress is linked to serious consequences such as depression, cardiovascular disease, weakened immunity, violent behavior, and even suicidal tendencies. These 
risks highlight the growing need for early detection, continuous monitoring, and timely intervention. Establishing reliable and efficient stress-detection frameworks can 
greatly improve overall well-being by helping individuals manage stress more effectively. Such systems can contribute to better academic performance, enhanced workplace 
productivity, and more responsive medical care, making stress detection an essential component of modern health monitoring solutions.


\subsection{Deep Learning Architectures for EEG Analysis}
maleesha ube kalla mekata dapan


\subsection{An Introduction to Hyperbolic Geometry for Machine Learning}
The vast majority of deep learning models operate in Euclidean space, the familiar geometry of flat surfaces. However, certain data types, particularly those with an 
underlying hierarchical or tree-like structure, can be embedded into Euclidean space only with significant distortion (Ganea et al., 2018). Hyperbolic geometry, a type 
of non-Euclidean geometry characterized by constant negative curvature, provides a compelling alternative. In hyperbolic space, the volume grows 
exponentially with the radius, mirroring the exponential growth of nodes in a tree. This property allows hierarchical structures to be embedded 
with much lower distortion, preserving their metric properties more faithfully (He et al., 2025).
A common and convenient model for implementing hyperbolic geometry is the Poincaré ball model. This model represents an n-dimensional hyperbolic 
space as the interior of an n-dimensional unit ball, equipped with a specialized metric that causes distances to expand infinitely as they approach 
the boundary (Peng et al., 2022). To perform computations and build neural networks within this space, standard vector operations must be replaced 
with their hyperbolic equivalents, which are defined within the framework of Möbius gyrovector spaces. Key operations include:
\begin{itemize}
\item Möbius addition: A non-commutative and non-associative generalization of vector addition.
\item Möbius scalar multiplication: A generalization of scalar-vector multiplication.
\end{itemize}
These principled generalizations, along with corresponding definitions for operations like matrix-vector multiplication, allow for the development of
hyperbolic analogues of standard neural network layers (Ganea et al., 2018). By defining fully connected, recurrent, and other layers that operate 
entirely within the Poincaré ball, we can build deep learning models that are geometrically tailored to the intrinsic structure of hierarchical data, 
setting the stage for their application to complex EEG signals.
counting.

\section{A General Pipeline for EEG-Based Stress Detection}
Research in EEG-based stress detection typically follows a standardized, end-to-end process, beginning with data collection and 
culminating in model evaluation. This systematic pipeline ensures that experiments are reproducible and provides a common framework 
for comparing the efficacy of different methodologies. The following sections describe the key stages of this process, from acquiring 
and cleaning the raw signals to deriving inputs for classification models

\subsection{Data Acquisition and Public Datasets}\label{SCM}

The first step in any machine learning project is acquiring high-quality data. In the field of EEG-based emotion and stress recognition, a 
number of publicly available datasets have become benchmarks for evaluating new algorithms. These include:
\begin{itemize}
\item DEAP (Dataset for Emotion Analysis using Physiological signals)
\item SEED and SEED-IV (SJTU Emotion EEG Dataset)
\item DREAMER (Database for Emotion Recognition through EEG and ECG)
\end{itemize}
These datasets typically involve recording multi-channel EEG signals (e.g., 32 or 62 channels) from participants while they are exposed
to stimuli designed to elicit specific emotional states, such as watching carefully selected music video clips (Li et al., 2023)


\subsection{Signal Preprocessing and Artifact Removal}\label{AAA}
Raw EEG data is invariably contaminated by noise and artifacts from both physiological and environmental sources. Therefore, a critical 
preprocessing stage is required to clean the signals before any analysis can be performed. Key steps include:

Band-pass filtering: This is applied to isolate the frequency range of interest and remove unwanted noise. A common range used in emotion 
recognition studies is 4-45 Hz, which effectively removes low-frequency physiological artifacts (e.g., from breathing or heartbeats) and 
high-frequency environmental noise (e.g., 50/60 Hz power line interference) (Islam et al., 2021).

Artifact Removal: Specific techniques are used to remove artifacts caused by muscle movements (electromyography, EMG) and eye blinks 
(electrooculography, EOG). Independent Component Analysis (ICA) is a powerful method for separating these artifactual sources from the 
underlying brain signals (Katmah et al., 2021). Artifact Subspace

Reconstruction (ASR) is another technique that identifies and statistically interpolates high-variance signal components that exceed a 
certain threshold (Li et al., 2023).

Re-referencing: The recorded EEG potentials are relative. To standardize the signals and reduce common noise across channels, the data is 
often re-referenced. The Common Average Reference (CAR) method, which subtracts the average value of all electrodes from each individual 
electrode, is a widely recommended approach (Li et al., 2023)

\subsection{Feature Extraction Methodologies}\label{ITH}
Once the EEG signals are preprocessed, the next step is to derive inputs suitable for a machine learning model. This is governed by one 
of two primary philosophies for feature handling, which are increasingly combined in hybrid approaches.

The traditional philosophy involves manually calculating a set of features from the signal, often based on domain knowledge. These 
"hand-crafted" features are designed to capture specific characteristics of the EEG signal across different domains.

Frequency-Domain Features: These describe the power distribution across different frequency bands
\begin{itemize}
\item Power Spectral Density (PSD): The average power of the signal in a specific frequency band (e.g., Alpha, Beta) (Islam et al., 2021).
\item Differential Entropy (DE): A feature related to the complexity of the signal within a frequency band, often used in emotion 
recognition (Li et al., 2023)
\end{itemize}
Time-Frequency Domain Features: These capture how the frequency content of the signal changes over time
\begin{itemize}
\item Discrete Wavelet Transform (DWT): Decomposes the signal into different frequency sub-bands using a mother wavelet (e.g., 'db4'), 
providing both time and frequency information (Nirabi et al., 2021)
\item oShort-Time Fourier Transform (STFT): Calculates the frequency spectrum over short, overlapping time windows (Li et al., 2023).
\end{itemize}
\subsection{Figures and Tables}\label{FAT}
\paragraph{Positioning Figures and Tables} Place figures and tables at the top and 
bottom of columns. Avoid placing them in the middle of columns. Large 
figures and tables may span across both columns. Figure captions should be 
below the figures; table heads should appear above the tables. Insert 
figures and tables after they are cited in the text. Use the abbreviation 
``Fig.~\ref{fig}'', even at the beginning of a sentence.

\begin{table}[htbp]
\caption{Table Type Styles}
\begin{center}
\begin{tabular}{|c|c|c|c|}
\hline
\textbf{Table}&\multicolumn{3}{|c|}{\textbf{Table Column Head}} \\
\cline{2-4} 
\textbf{Head} & \textbf{\textit{Table column subhead}}& \textbf{\textit{Subhead}}& \textbf{\textit{Subhead}} \\
\hline
copy& More table copy$^{\mathrm{a}}$& &  \\
\hline
\multicolumn{4}{l}{$^{\mathrm{a}}$Sample of a Table footnote.}
\end{tabular}
\label{tab1}
\end{center}
\end{table}

\begin{figure}[htbp]
\centerline{\includegraphics{fig1.png}}
\caption{Example of a figure caption.}
\label{fig}
\end{figure}

Figure Labels: Use 8 point Times New Roman for Figure labels. Use words 
rather than symbols or abbreviations when writing Figure axis labels to 
avoid confusing the reader. As an example, write the quantity 
``Magnetization'', or ``Magnetization, M'', not just ``M''. If including 
units in the label, present them within parentheses. Do not label axes only 
with units. In the example, write ``Magnetization (A/m)'' or ``Magnetization 
\{A[m(1)]\}'', not just ``A/m''. Do not label axes with a ratio of 
quantities and units. For example, write ``Temperature (K)'', not 
``Temperature/K''.

\section*{Acknowledgment}

The preferred spelling of the word ``acknowledgment'' in America is without 
an ``e'' after the ``g''. Avoid the stilted expression ``one of us (R. B. 
G.) thanks $\ldots$''. Instead, try ``R. B. G. thanks$\ldots$''. Put sponsor 
acknowledgments in the unnumbered footnote on the first page.

\section*{References}

Please number citations consecutively within brackets \cite{b1}. The 
sentence punctuation follows the bracket \cite{b2}. Refer simply to the reference 
number, as in \cite{b3}---do not use ``Ref. \cite{b3}'' or ``reference \cite{b3}'' except at 
the beginning of a sentence: ``Reference \cite{b3} was the first $\ldots$''

Number footnotes separately in superscripts. Place the actual footnote at 
the bottom of the column in which it was cited. Do not put footnotes in the 
abstract or reference list. Use letters for table footnotes.

Unless there are six authors or more give all authors' names; do not use 
``et al.''. Papers that have not been published, even if they have been 
submitted for publication, should be cited as ``unpublished'' \cite{b4}. Papers 
that have been accepted for publication should be cited as ``in press'' \cite{b5}. 
Capitalize only the first word in a paper title, except for proper nouns and 
element symbols.

For papers published in translation journals, please give the English 
citation first, followed by the original foreign-language citation \cite{b6}.
\bibliographystyle{IEEEtran}
\bibliography{references}


\vspace{12pt}
\color{red}
IEEE conference templates contain guidance text for composing and formatting conference papers. Please ensure that all template text is removed from your conference paper prior to submission to the conference. Failure to remove the template text from your paper may result in your paper not being published.

\end{document}
